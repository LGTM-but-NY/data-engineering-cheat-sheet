\documentclass[landscape]{article}
\sloppy
\usepackage{amsmath}
\usepackage{amssymb}
\usepackage{graphicx}
\usepackage{hyperref}
\usepackage{geometry}
\usepackage{multicol} % for multi-column layout
\usepackage{enumitem} % to tighten list spacing

% --- VIETNAMESE SUPPORT PACKAGES ---
\usepackage[utf8]{inputenc}
\usepackage[T5]{fontenc}
\usepackage[vietnamese]{babel}
% -----------------------------------

% Tight margins for cheat sheet in landscape (A4 297mm x 210mm)
\geometry{a4paper, landscape, left=8mm, right=8mm, top=8mm, bottom=8mm}

\title{Tổng hợp Kiến thức Kỹ thuật Dữ liệu (Data Engineering)}
\author{Học viên Cao học KHMT Bách Khoa TP.HCM (HCMUT)}
\date{\today}

\begin{document}

\maketitle
\footnotesize % smaller font for dense content
% Zero / minimal spacing in itemize and enumerate lists
\setlist{itemsep=0pt,topsep=2pt,parsep=0pt,partopsep=0pt}
\begin{multicols}{3}
    \setlength{\columnsep}{8pt} % space between columns
    \setlength{\multicolsep}{2pt} % vertical space before/after multicols
    \raggedcolumns
    \section{Nguyên lý Phân tích \& Thiết kế CSDL}

    \subsection{Tổng quan Các giai đoạn Thiết kế}
    \begin{enumerate}[leftmargin=*,itemsep=2pt]
        \item \textbf{Mức Quan niệm (Conceptual):} \textit{Mục tiêu} — nắm bắt yêu cầu và ngữ nghĩa (độc lập với cài đặt); \textit{Mô hình/Công cụ} — ER/EER; \textit{Đầu ra} — lược đồ quan niệm (thực thể, thuộc tính, mối kết hợp, ràng buộc).
        \item \textbf{Mức Logic:} \textit{Mục tiêu} — ánh xạ từ mức quan niệm sang mô hình DBMS đích (ví dụ: quan hệ); \textit{Mô hình/Công cụ} — ánh xạ ER-sang-quan hệ, chuẩn hóa (FDs); \textit{Đầu ra} — lược đồ quan hệ (bảng, khóa, ràng buộc toàn vẹn).
        \item \textbf{Mức Vật lý:} \textit{Mục tiêu} — xác định cấu trúc lưu trữ và đường dẫn truy xuất để tối ưu hiệu năng; \textit{Mô hình/Công cụ} — phân tích tải, chỉ mục, tổ chức tập tin, băm; \textit{Đầu ra} — lược đồ trong (cấu trúc lưu trữ, chỉ mục, đường dẫn truy xuất).
    \end{enumerate}

    \subsection{Nguyên lý Thiết kế Mức Quan niệm}
    \begin{itemize}[leftmargin=*, label={--}]
        \item \textbf{Phân tích Yêu cầu:} Làm việc với người dùng/chuyên gia nghiệp vụ để nắm bắt yêu cầu dữ liệu và yêu cầu chức năng (thao tác/giao dịch).
        \item \textbf{Thành phần ER:} \textbf{Thực thể} (vd: NHANVIEN), \textbf{Thuộc tính} (đơn/phức hợp/đa trị/dẫn xuất), \textbf{Mối kết hợp} (sự liên kết giữa các thực thể).
        \item \textbf{Ràng buộc Cấu trúc:} Tỷ số bản số (1:1, 1:N, M:N) và ràng buộc tham gia (\textit{toàn phần} vs \textit{từng phần}).
        \item \textbf{Thực thể Yếu:} Được xác định thông qua \textbf{mối kết hợp xác định} với một thực thể \textbf{chủ} và một \textbf{khóa bộ phận}; thực thể yếu tham gia \textit{toàn phần} vào mối kết hợp xác định.
        \item \textbf{Tinh chỉnh Top-Down:} Tinh chỉnh lặp lại các thực thể tổng quát; áp dụng chuyên biệt hóa/tổng quát hóa (EER).
    \end{itemize}

    \subsection{Nguyên lý Thiết kế Mức Logic}
    \subsubsection{Cơ bản về Mô hình Quan hệ}
    \begin{itemize}[leftmargin=*, label={--}]
        \item \textbf{Cấu trúc:} Lược đồ quan hệ $R(A_1, \dots, A_n)$; các bộ (tuple) không có thứ tự và không cho phép trùng lặp trong mô hình hình thức.
        \item \textbf{Ràng buộc Toàn vẹn:} Ràng buộc miền giá trị (nguyên tố, có kiểu), ràng buộc khóa (siêu khóa/khóa ứng viên/khóa chính), toàn vẹn thực thể (khóa chính không NULL), toàn vẹn tham chiếu (giá trị khóa ngoại phải xuất hiện trong khóa chính được tham chiếu).
    \end{itemize}
    \subsubsection{Lý thuyết Chuẩn hóa}
    \begin{itemize}[leftmargin=*, label={--}]
        \item \textbf{Phụ thuộc Hàm (FD):} $X \rightarrow Y$ nghĩa là các bộ giống nhau trên $X$ thì phải giống nhau trên $Y$.
        \item \textbf{Các tính chất Thiết kế:}
        \begin{enumerate}[leftmargin=*]
            \item \textbf{Kết nối bảo toàn thông tin (Không tổn thất):} Phân rã không được tạo ra các bộ giả (spurious tuples).
            \item \textbf{Bảo toàn phụ thuộc:} Các FD ban đầu có thể kiểm tra được trên các quan hệ đã phân rã.
            \item \textbf{Giảm thiểu dư thừa:} Tránh các dị thường cập nhật (thêm, xóa, sửa).
        \end{enumerate}
        \item \textbf{Các Dạng chuẩn:} 1NF (nguyên tố), 2NF (FD đầy đủ vào khóa), 3NF (không có FD bắc cầu), BCNF (với mọi FD $X \rightarrow A$, $X$ là siêu khóa; có thể không bảo toàn phụ thuộc). 4NF/5NF giải quyết phụ thuộc đa trị và phụ thuộc kết nối.
        \item \textbf{Phi chuẩn hóa (Denormalization):} Lưu trữ kết quả kết nối như quan hệ cơ sở để tăng hiệu năng nhưng chấp nhận dị thường.
    \end{itemize}

    \subsection{Nguyên lý Thiết kế Mức Vật lý}
    \begin{itemize}[leftmargin=*, label={--}]
        \item \textbf{Kiến trúc Lưu trữ:} Dữ liệu bền vững trên đĩa/SSD trong các \textbf{khối (block)} kích thước cố định.
        \item \textbf{Phân tích Tải (Job Mix):} Xác định các quan hệ/tập tin thường truy cập, điều kiện chọn (bằng/khác/khoảng), và tần suất cập nhật so với truy vấn.
        \item \textbf{Cấu trúc Chỉ mục:} Chỉ mục có thứ tự (B+-Trees) và chỉ mục băm; chỉ mục \textbf{chính/phân cụm} (quy định thứ tự vật lý; tối đa một trên mỗi tập tin) so với chỉ mục \textbf{phụ}.
        \item \textbf{Tối ưu hóa Truy vấn:} Dựa trên chi phí (thống kê) và các quy tắc kinh nghiệm (đẩy phép chọn/chiếu xuống sớm) để chọn kế hoạch hiệu quả.
    \end{itemize}

    \section{Lưu trữ Dữ liệu \& Chỉ mục}

    % ============================================================================
    % PART 1: FOUNDATIONAL CONCEPTS
    % ============================================================================
    \subsection{Cơ bản về Chỉ mục}

    \subsubsection{Chỉ mục là gì?}
    \begin{itemize}[leftmargin=*, label={--}]
        \item \textbf{Mục đích:} Tăng tốc độ truy xuất dữ liệu bằng cách tạo đường dẫn phụ trợ đến các bản ghi.
        \item \textbf{Khóa tìm kiếm:} (Các) thuộc tính dùng để tìm bản ghi; không nhất thiết là khóa chính; có thể là khóa phức hợp (nhiều cột).
        \item \textbf{Đánh đổi:} Đọc nhanh hơn so với ghi chậm hơn (chi phí bảo trì chỉ mục khi INSERT/UPDATE/DELETE).
    \end{itemize}

    \subsubsection{Phân loại Chỉ mục}
    \begin{itemize}[leftmargin=*, label={--}]
        \item \textbf{Theo Cấu trúc:}
        \begin{itemize}
            \item \textit{Có thứ tự (Ordered):} Các mục được sắp xếp (vd: B+-tree); hỗ trợ truy vấn khoảng.
            \item \textit{Băm (Hash):} Khóa được băm vào bucket; chỉ nhanh khi tìm kiếm chính xác (dấu bằng).
        \end{itemize}
        \item \textbf{Theo Mật độ:}
        \begin{itemize}
            \item \textit{Đặc (Dense):} Một mục chỉ mục cho mỗi giá trị khóa tìm kiếm riêng biệt.
            \item \textit{Thưa (Sparse):} Một mục chỉ mục cho mỗi khối (hoặc mỗi giá trị phân cụm); nhỏ hơn, chi phí bảo trì thấp hơn.
        \end{itemize}
        \item \textbf{Theo Thứ tự Vật lý:}
        \begin{itemize}
            \item \textit{Chỉ mục Chính/Phân cụm:} Khóa tìm kiếm quyết định thứ tự vật lý của tập tin (tối đa một per bảng); thường là chỉ mục thưa.
            \item \textit{Chỉ mục Phụ:} Đường dẫn truy cập thay thế độc lập với thứ tự vật lý; thường là chỉ mục đặc hoặc dùng gián tiếp cho các khóa không duy nhất.
        \end{itemize}
    \end{itemize}

    % ============================================================================
    % PART 2: CORE INDEX STRUCTURES
    % ============================================================================
    \subsection{B-Trees \& B+-Trees}

    \subsubsection{Tại sao dùng B+-Trees?}
    \begin{itemize}[leftmargin=*, label={--}]
        \item \textbf{Mục tiêu:} Giảm thiểu I/O đĩa tốn kém bằng cách giữ chiều cao cây thấp thông qua hệ số rẽ nhánh (fan-out) cao.
        \item \textbf{Cấu trúc:} Cây cân bằng; nút trong chứa khóa + con trỏ con; nút lá chứa khóa + con trỏ dữ liệu (hoặc ID bản ghi) + liên kết đến lá kế tiếp.
        \item \textbf{Thao tác:} Tìm/thêm/xóa trong $O(\log_{fo} n)$ lần truy cập khối; tách/gộp nút để duy trì cân bằng.
        \item \textbf{B+- so với B-Tree:} B+- chỉ lưu con trỏ dữ liệu ở lá $\Rightarrow$ hệ số rẽ nhánh nút trong cao hơn, quét tuần tự hiệu quả qua chuỗi liên kết lá.
    \end{itemize}

    \subsubsection{Ví dụ Tính Dung lượng}
    \textit{Tham số:} Kích thước khối $B=512$ bytes; kích thước khóa $V=9$ bytes; con trỏ dữ liệu $\Pr=7$ bytes; con trỏ cây $P=6$ bytes.

    \paragraph{Bước 1: Tính Bậc (Số con trỏ tối đa mỗi nút)}
    \begin{itemize}[leftmargin=*, label={--}]
        \item \textbf{Nút trong B-Tree:} Chứa $p$ con trỏ cây + $(p-1)$ khóa + $(p-1)$ con trỏ dữ liệu.
        \item[] $(p \times 6) + ((p-1) \times (7+9)) \le 512$
        \item[] $6p + 16p - 16 \le 512 \quad \Rightarrow \quad 22p \le 528 \quad \Rightarrow \quad p = 23$
        
        \item \textbf{Nút trong B+-Tree:} Chứa $p$ con trỏ cây + $(p-1)$ khóa (không có con trỏ dữ liệu).
        \item[] $(p \times 6) + ((p-1) \times 9) \le 512$
        \item[] $6p + 9p - 9 \le 512 \quad \Rightarrow \quad 15p \le 521 \quad \Rightarrow \quad p = 34$
        
        \item \textbf{Nút lá B+-Tree:} Chứa $p_{leaf}$ cặp khóa/con trỏ dữ liệu + 1 con trỏ kế tiếp.
        \item[] $(p_{leaf} \times (7+9)) + 6 \le 512$
        \item[] $16 \times p_{leaf} \le 506 \quad \Rightarrow \quad p_{leaf} = 31$
    \end{itemize}

    \paragraph{Bước 2: Ước lượng Tổng dung lượng (Đầy 69\%)}
    \begin{itemize}[leftmargin=*, label={--}]
        \item \textbf{B-Tree (3 mức):} Hệ số rẽ nhánh trung bình $fo = 23 \times 0.69 \approx 16$.
        \begin{itemize}
            \item Mức 0 (gốc): 15 mục, 16 con trỏ
            \item Mức 1: $16 \times 15 = 240$ mục
            \item Mức 2: $256 \times 15 = 3{,}840$ mục
            \item \textbf{Tổng:} $15 + 240 + 3{,}840 \approx \mathbf{4{,}095}$ mục
        \end{itemize}
        
        \item \textbf{B+-Tree (3 mức):} Nút trong $fo = 34 \times 0.69 \approx 23$; Dung lượng lá $= 31 \times 0.69 \approx 21$.
        \begin{itemize}
            \item Mức 0: 22 mục, 23 con trỏ
            \item Mức 1: $23 \times 22 = 506$ mục, 529 con trỏ
            \item Mức lá: $12{,}167 \times 21 \approx \mathbf{255{,}507}$ con trỏ dữ liệu
        \end{itemize}
        
        \item \textbf{Nhận xét:} B+- chứa được $\sim 4\times$ số mục ở cùng chiều cao nhờ nút trong nhẹ hơn.
    \end{itemize}

    \subsubsection{Các loại Chỉ mục Nâng cao}
    \begin{itemize}[leftmargin=*, label={--}]
        \item \textbf{Chỉ mục Phức hợp:} Khóa nhiều cột (vd: (City, LastName)); hỗ trợ truy vấn tiền tố trái nhất (City), (City, LastName); sắp xếp cột theo độ chọn lọc.
        \item \textbf{Chỉ mục Dựa trên Hàm:} Chỉ mục trên biểu thức (vd: \texttt{LOWER(email)}); truy vấn phải dùng đúng hàm đó mới tận dụng được.
    \end{itemize}

    % ============================================================================
    % PART 3: SPECIALIZED STRUCTURES
    % ============================================================================
    \subsection{Chỉ mục Băm \& Bitmap}

    \subsubsection{Chỉ mục Băm (Hash Indexes)}
    \begin{itemize}[leftmargin=*, label={--}]
        \item \textbf{Sử dụng:} Tìm kiếm chính xác cực nhanh (truy vấn điểm); không hỗ trợ khoảng.
        \item \textbf{Xử lý đụng độ:} Dùng danh sách liên kết (chaining) với bucket tràn.
        \item \textbf{Biến thể động:} Băm mở rộng/tuyến tính (Extendible/Linear hashing) tăng trưởng dần mà không cần xây lại toàn bộ.
    \end{itemize}

    \subsubsection{Chỉ mục Bitmap}
    \textit{Tối ưu cho thuộc tính có độ chọn lọc thấp (ít giá trị riêng biệt).}
    \begin{itemize}[leftmargin=*, label={--}]
        \item \textbf{Cấu trúc:} Đánh số thứ tự bản ghi (0, 1, 2, ...); mỗi giá trị riêng biệt có một bitmap; bit $i=1$ nếu bản ghi $i$ có giá trị đó.
        \item \textbf{Ví dụ (bảng 5 dòng):}
        \begin{itemize}
            \item \texttt{gender='m'}: 10010 \quad \texttt{gender='f'}: 01101
            \item \texttt{income='L1'}: 11000 \quad \texttt{income='L2'}: 00100
        \end{itemize}
        \item \textbf{Truy vấn: gender='f' AND income='L2'}
        \item[] $01101 \; \text{AND} \; 00100 = 00100 \quad \Rightarrow \quad \text{bản ghi 2}$
        \item \textbf{Ưu điểm:} Gọn nhẹ (1 triệu dòng = 125 KB mỗi bitmap); thao tác bitwise nhanh; hiệu quả cho bộ lọc nhiều điều kiện; hỗ trợ COUNT qua đếm bit.
    \end{itemize}

    % ============================================================================
    % PART 4: QUERY COST MODELING
    % ============================================================================
    \subsection{Tối ưu hóa Truy vấn \& Phân tích Chi phí}

    \subsubsection{Đo lường Chi phí \& Thông tin Catalog}
    \begin{itemize}[leftmargin=*, label={--}]
        \item \textbf{Độ đo chính (I/O):} Giảm thiểu chuyển khối ($b$) và truy cập ngẫu nhiên (seek $S$). Thời gian: $b \times t_T + S \times t_S$.
        \item \textbf{Kích thước quan hệ:} $r$ = số bộ, $b$ = số khối tập tin.
        \item \textbf{Chi tiết chỉ mục:} $x$ = chiều cao chỉ mục đa mức (vd: B+-tree), $b_{I1}$ = số khối chỉ mục mức 1.
        \item \textbf{Hệ số khối (Blocking factor):} $bfr$ = số bộ trên mỗi khối.
        \item \textbf{Số lượng chọn (Selection cardinality):} $s = sl \times r$ với $sl$ là độ chọn lọc (selectivity).
        \item \textbf{Số giá trị phân biệt:} $NDV(A)$ = số giá trị khác nhau của thuộc tính $A$.
    \end{itemize}

    \subsubsection{Hàm Chi phí cho Phép Chọn}
        \textit{Phép chọn ($\sigma$) có thể dùng quét tập tin hoặc truy cập chỉ mục/băm tùy đường dẫn có sẵn. Chi phí chưa tính việc ghi kết quả.}
    \paragraph{\textbf{S1: Tìm kiếm Tuyến tính (Vét cạn / A1)}}
    \begin{itemize}[leftmargin=*, label={--}]
        \item Trường hợp xấu nhất/không khóa: $\mathbf{C_{S1a} = b}$.
        \item Trung bình tìm bằng trên khóa: $\mathbf{C_{S1b} = b/2}$ (dừng khi thấy).
    \end{itemize}
    \paragraph{\textbf{S2: Tìm kiếm Nhị phân (tập tin có thứ tự)}}\mbox{}\\
     $\mathbf{C_{S2} = \log_2 b + \lceil s/bfr \rceil - 1}$.
    \paragraph{\textbf{S3a: Chỉ mục Chính (một bản ghi)}}\mbox{}\\
     $\mathbf{C_{S3a} = x + 1}$.
    \paragraph{\textbf{S3b: Khóa Băm (một bản ghi)}}\mbox{}\\
     $\mathbf{C_{S3b} = 1}$ (tĩnh/tuyến tính) hoặc $\mathbf{2}$ (mở rộng).
    \paragraph{\textbf{S5: Chỉ mục Phân cụm (bằng trên không khóa / A3)}}\mbox{}\\
     $\mathbf{C_{S5} = x + \lceil s/bfr \rceil}$.
    \paragraph{\textbf{S6a: Chỉ mục Phụ (bằng trên không khóa / A4)}}\mbox{}\\
     $\mathbf{C_{S6a} = x + 1 + s}$ (trường hợp xấu nhất, bản ghi phân tán).
    \paragraph{\textbf{S6b: Chỉ mục Phụ (truy vấn khoảng)}}\mbox{}\\
     $\mathbf{C_{S6b} = x + (b_{I1}/2) + (r/2)}$.\\
    
    \textit{Lưu ý: Chi phí thời gian thường mô hình hóa là $b \times t_T + S \times t_S$, tách biệt truyền dữ liệu và tìm kiếm đầu từ.}

    \subsubsection{Ví dụ: Phép chọn trên EMPLOYEE}
        \textit{Kịch bản:} EMPLOYEE có $r_E=10{,}000$, $b_E=2{,}000$, $bfr_E=5$. Các chỉ mục/đường dẫn có sẵn:
    \begin{itemize}
        \item \textbf{Salary} (phân cụm, không khóa): $x=3$, $s_{Salary}=20$.
        \item \textbf{Ssn} (phụ, khóa): $x=4$, $s_{Ssn}=1$.
        \item \textbf{Dno} (phụ, không khóa): $x=2$, $s_{Dno}=80$ (từ $10{,}000/125$).
        \item \textbf{Sex} (phụ, không khóa): $x=1$, $s_{Sex}=5{,}000$ (từ $10{,}000/2$).
    \end{itemize}

    \paragraph{OP1: Tìm bằng trên Khóa}\mbox{}\\
    Truy vấn: $\sigma_{\text{Ssn}='123456789'}(\text{EMPLOYEE})$.
    \begin{itemize}[leftmargin=*, label={--}]
        \item S1b (tuyến tính tb): $C_{S1b} = b_E/2 = 1{,}000$.
        \item S6a (chỉ mục phụ trên khóa): $C_{S6a} = x_{Ssn} + 1 = 4 + 1 = \mathbf{5}$.
        \item \textbf{Quyết định:} Chọn S6a (5 $\ll$ 1{,}000).
    \end{itemize}

    \paragraph{OP3: Tìm bằng trên Không khóa}\mbox{}\\
    Truy vấn: $\sigma_{\text{Dno}=5}(\text{EMPLOYEE})$.
    \begin{itemize}[leftmargin=*, label={--}]
        \item S1a (tuyến tính): $C_{S1a} = b_E = \mathbf{2{,}000}$.
        \item S6a (phụ trên Dno): $C_{S6a} = x_{Dno} + s_{Dno} = 2 + 80 = \mathbf{82}$.
        \item \textbf{Quyết định:} Chọn S6a (82 $\ll$ 2{,}000). Nếu có chỉ mục phân cụm trên Dno: $3 + \lceil 80/5 \rceil = 19$ khối.
    \end{itemize}

    \paragraph{OP4: Phép chọn Hội (Nhiều điều kiện)}\mbox{}\\
    Truy vấn: $\sigma_{\text{Dno}=5 \land \text{Salary}>30\,000 \land \text{Sex}='F'}(\text{EMPLOYEE})$.
    Bộ tối ưu so sánh các đường dẫn truy cập để lấy tập ứng viên ban đầu, sau đó kiểm tra các vị từ còn lại trong RAM.
    \begin{itemize}[leftmargin=*, label={--}]
        \item Qua Dno (S6a): $C = x_{Dno} + s_{Dno} = \mathbf{82}$.
        \item Qua khoảng Salary (phân cụm): $C \approx x_{Salary} + (b_E/2) = 3 + 1{,}000 = \mathbf{1{,}003}$.
        \item Qua Sex (S6a): $C = x_{Sex} + s_{Sex} = 1 + 5{,}000 = \mathbf{5{,}001}$.
        \item Vét cạn (S1a): $C = 2{,}000$.
        \item \textbf{Quyết định:} Dùng chỉ mục Dno (82), lấy 80 bộ, sau đó lọc $\text{Salary}>30\,000$ và $\text{Sex}='F'$ trong RAM.
    \end{itemize}

    \subsubsection{Thuật toán Kết nối (Join) \& So sánh Chi phí}

    \paragraph{Tham số Chính}
    \begin{itemize}[leftmargin=*, label={--}]
        \item \textbf{Độ chọn lọc kết nối:} $js = |R \bowtie S|/(|R|\,|S|)$; với equi-join $js \approx 1/\max(NDV(A), NDV(B))$.
        \item \textbf{Số bộ kết quả:} $jc = js\,|R|\,|S|$; \textbf{chi phí ghi}: $jc/bfr_{result}$ khối.
        \item \textbf{Bộ đệm:} $n_B$ = số trang đệm khả dụng (ảnh hưởng chi phí nested-loop).
    \end{itemize}

    \paragraph{Kịch bản Thống nhất}\mbox{}\\
    EMPLOYEE ($|E|=10{,}000$, $b_E=2{,}000$) $\bowtie_{Dno=Dnumber}$ \\
    DEPARTMENT ($|D|=125$, $b_D=13$). \\
    Chỉ mục trên $E.Dno$ (phụ: $x=2$, $s=80$).\\
    $Dnumber$ là khóa chính ($x=1$).\\
    Giả sử $js=1/125$, $jc=10{,}000$, $bfr_{result}=4$ (chi phí ghi $=2{,}500$ khối), $n_B=3$.

    \paragraph{J1: Block Nested-Loop}\mbox{}\\
    Chi phí: $C_{J1} = b_R + \left\lceil \frac{b_R}{n_B - 2} \right\rceil b_S + \frac{jc}{bfr_{result}}$.
    Dùng DEPARTMENT làm vòng ngoài: $C_{J1} = 13 + \lceil 13/1\rceil \times 2{,}000 + 2{,}500 = \mathbf{28{,}513}$.

    \paragraph{J2: Indexed Nested-Loop}\mbox{}\\
    \begin{itemize}[leftmargin=*, label={--}]
        \item \textbf{DEPARTMENT ngoài $\to$ EMPLOYEE trong:} Mỗi lần tìm $= x+s = 2+80 = 82$. Tổng $= 13 + 125 \times 82 + 2{,}500 = \mathbf{12{,}763}$.
        \item \textbf{EMPLOYEE ngoài $\to$ DEPARTMENT trong:} Mỗi lần tìm $= x+1 = 1+1 = 2$. Tổng $= 2{,}000 + 10{,}000 \times 2 + 2{,}500 = \mathbf{24{,}500}$.
    \end{itemize}

    \paragraph{J3: Sort-Merge (Trộn sắp xếp)}\mbox{}\\
    Nếu đã sắp xếp: \[C_{J3} = b_E + b_D + \frac{jc}{bfr_{result}} = 2{,}000 + 13 + 2{,}500 = \mathbf{4{,}513}\]. Nếu chưa, cộng thêm chi phí sắp xếp ngoài mỗi quan hệ.

    \paragraph{J4: Partition-Hash (Băm phân hoạch)}\mbox{}\\
    Chi phí xấp xỉ: $C_{J4} \approx 3\,(b_E + b_D) + \frac{jc}{bfr_{result}} = 3\times(2{,}000+13) + 2{,}500 = \mathbf{8{,}539}$.

    \paragraph{Quyết định Kế hoạch}\mbox{}\\
    J3 (nếu đã sắp xếp) $<$ J4 $<$ J2\,(D ngoài) $<$ J2\,(E ngoài) $<$ J1. Ưu tiên Hash Join khi chưa sắp xếp; ưu tiên Sort-Merge khi đã có thứ tự.

    % ============================================================================
    % PART 5: OPTIMIZATION STRATEGIES
    % ============================================================================
    \subsection{Quy trình Tối ưu hóa Truy vấn}

    \subsubsection{Tổng quan Quy trình}
    \begin{enumerate}[leftmargin=*]
        \item \textbf{Phân tích \& Kiểm tra (Parse):} Kiểm tra cú pháp, tuân thủ lược đồ.
        \item \textbf{Dịch (Translate):} Chuyển SQL sang đại số quan hệ (cây truy vấn).
        \item \textbf{Tối ưu hóa Kinh nghiệm (Heuristic):} Áp dụng các quy tắc biến đổi.
        \item \textbf{Tối ưu hóa Dựa trên Chi phí:} Liệt kê các kế hoạch, ước tính chi phí, chọn chi phí thấp nhất.
        \item \textbf{Thực thi:} Hiện thực hóa kết quả trung gian hoặc pipeline kết quả.
    \end{enumerate}

    \subsubsection{Quy tắc Heuristic}
    \begin{itemize}[leftmargin=*, label={--}]
        \item \textbf{Đẩy phép chọn ($\sigma$) xuống:} Lọc sớm để giảm kích thước trung gian.
        \item \textbf{Đẩy phép chiếu ($\Pi$) xuống:} Giảm chiều rộng bộ dữ liệu sớm.
        \item \textbf{Thay thế $\sigma$ + Tích đề-các ($\times$) bằng Kết ($\bowtie$):} Tránh tích đề-các tốn kém.
        \item \textbf{Sắp xếp lại thứ tự kết:} Dùng tính giao hoán/kết hợp để tìm kết quả trung gian có lực lượng thấp.
        \item \textbf{Cây nghiêng trái (Left-deep trees):} Con phải luôn là bảng cơ sở $\Rightarrow$ cho phép tìm kiếm chỉ mục, giảm không gian tìm kiếm.
    \end{itemize}

    \subsubsection{Quyết định Dựa trên Chi phí}
    \begin{itemize}[leftmargin=*, label={--}]
        \item \textbf{Chọn đường dẫn truy cập:} So sánh quét toàn bộ, chỉ mục phân cụm, chỉ mục phụ, bitmap cho mỗi vị từ.
        \item \textbf{Thuật toán kết nối:} Nested-loop (index/block), hash join, sort-merge dựa trên số lượng bộ và bộ nhớ.
        \item \textbf{Thứ tự kết nối:} Ước lượng độ chọn lọc kết nối $js \approx 1/\max(NDV(A), NDV(B))$; số lượng $jc = js \times |R| \times |S|$.
    \end{itemize}

    \subsection{Kỹ thuật Chuyên biệt}

    \subsubsection{Cấu trúc Tối ưu Ghi}
    \begin{itemize}[leftmargin=*, label={--}]
        \item \textbf{LSM-Tree:} Bộ đệm trong RAM (memtable) + các mức đĩa đã sắp xếp; ghi tuần tự; nén định kỳ; Bloom filter để bỏ qua các mức.
        \item \textbf{Buffer Tree:} Biến thể B-tree với bộ đệm ghi tại mỗi nút; gom nhóm các thay đổi xuống cây; độ trễ đọc tốt hơn LSM.
    \end{itemize}

    \subsubsection{Truy cập Không gian \& Đa khóa}
    \begin{itemize}[leftmargin=*, label={--}]
        \item \textbf{Đa khóa:} Chỉ mục phức hợp cho truy vấn tiền tố; chỉ mục bao phủ (covering index) tránh tra cứu bảng.
        \item \textbf{Không gian:} R-tree (hình chữ nhật bao), kd-tree, quadtree cho dữ liệu địa lý/khoảng; hỗ trợ truy vấn vùng + lân cận gần nhất.
    \end{itemize}

    % ============================================================================
    % PART 6: ADVANCED TOPICS
    % ============================================================================
    \section{Lưu trữ Dữ liệu Lớn}

    \subsection{Khái niệm}
    \begin{itemize}[leftmargin=*, label={--}]
        \item \textbf{Động lực NoSQL:} Khả năng mở rộng ngang, tính sẵn sàng, lược đồ linh hoạt (BASE/CAP) so với ACID chặt chẽ.
        \item \textbf{3V của Big Data:} Volume (TB–PB–EB), Velocity (thời gian thực), Variety (cấu trúc/bán cấu trúc/phi cấu trúc); thường thêm Veracity (độ tin cậy) và Value (giá trị).
        \item \textbf{Chiến lược lược đồ:} \textit{Lược đồ khi ghi (Schema-on-write)} (DW) so với \textit{Lược đồ khi đọc (Schema-on-read)} (DL).
        \item \textbf{Lưu trữ phân tán:} Hệ thống kiểu HDFS cung cấp lưu trữ chịu lỗi, mở rộng trên các cụm máy.
    \end{itemize}

        \subsubsection{Động lực NoSQL (BASE/CAP vs ACID)}
        \begin{itemize}[leftmargin=*, label={--}]
            \item \textbf{ACID vs BASE:} RDBMS tuân thủ ACID; NoSQL thường theo BASE (Basically Available, Soft state, Eventually consistent) để mở rộng.
            \item \textbf{Đánh đổi CAP:} Trong hệ thống phân tán có sao chép, không thể đồng thời có Consistency (Nhất quán), Availability (Sẵn sàng), và Partition tolerance (Chịu phân hoạch). NoSQL thường chọn AP hơn C chặt chẽ.
            \item \textbf{Nhất quán cuối cùng (Eventual consistency):} Các bản sao có thể lệch nhau tạm thời nhưng sẽ hội tụ nếu không có cập nhật mới; chấp nhận được cho nhiều ứng dụng web.
            \item \textbf{Mở rộng ngang (Scale-out):} Mở rộng qua phân mảnh (sharding) và sao chép (replication) trên các node phổ thông.
            \item \textbf{Lược đồ linh hoạt:} Bản ghi tự mô tả (JSON/BSON), bán cấu trúc, thuộc tính không đồng nhất.
            \item \textbf{Mô hình:} Văn bản (MongoDB), Khóa-Giá trị (Redis/DynamoDB), Cột rộng (BigTable/HBase/Cassandra), Đồ thị (Neo4j).
        \end{itemize}

        \subsubsection{5 chữ V của Big Data}
        \begin{itemize}[leftmargin=*, label={--}]
            \item \textbf{Volume (Dung lượng):} TB đến PB đến EB; từ log, di động, giao dịch, cảm biến IoT; cần song song hóa lớn.
            \item \textbf{Velocity (Tốc độ):} Tốc độ nạp cao và xử lý luồng/thời gian thực để phát hiện gian lận, giám sát.
            \item \textbf{Variety (Đa dạng):} Cấu trúc, bán cấu trúc, phi cấu trúc (web, mạng xã hội, vị trí, ảnh, video, log). Dữ liệu phi cấu trúc là thách thức lớn.
            \item \textbf{Veracity (Độ xác thực):} Độ tin cậy/chất lượng dữ liệu biến thiên; cần kiểm tra trước khi phân tích.
            \item \textbf{Value (Giá trị):} Phân tích (mô tả/dự đoán/đề xuất) để tạo ra lợi ích kinh doanh.
        \end{itemize}

        \subsubsection{Lưu trữ Phân tán (Kiểu HDFS)}
        \begin{itemize}[leftmargin=*, label={--}]
            \item \textbf{Không chia sẻ (Shared-nothing):} Mỗi node có CPU/RAM/đĩa riêng; phối hợp qua mạng; mở rộng kinh tế nhờ dư thừa.
            \item \textbf{Kiến trúc HDFS:} NameNode quản lý không gian tên/metadata; DataNode lưu khối và phục vụ I/O; cụm có thể có hàng ngàn DataNode.
            \item \textbf{Sao chép (Replication):} Khối được sao chép (thường 3x) để bền vững và sẵn sàng; client đọc từ bản sao gần nhất.
            \item \textbf{I/O Song song:} Nhiều máy đọc/ghi đồng thời, tăng tổng thông lượng.
            \item \textbf{Chỉ thêm (Append-only):} Mô hình nhất quán đơn giản tối ưu cho batch: tập tin chỉ được thêm vào cuối (không update ngẫu nhiên).
            \item \textbf{Hệ sinh thái:} HDFS làm nền tảng cho MapReduce, HBase, và các công cụ big data khác.
        \end{itemize}

    \subsection{Các Công nghệ}
    \paragraph{Document Store (MongoDB)}
    \begin{itemize}[leftmargin=*, label={--}]
        \item \textit{Lược đồ khi đọc} linh hoạt cho tính Đa dạng (Variety); thuộc tính không đồng nhất.
        \item Tài liệu JSON/BSON (mảng, đối tượng lồng); thiết kế phi chuẩn hóa để tăng tính cục bộ dữ liệu.
        \item CRUD qua \texttt{find(<cond>)}; tự động đánh chỉ mục \_id để truy xuất theo khóa.
        \item Tính sẵn sàng cao qua Replica Sets (đọc primary/secondary).
        \item Mở rộng ngang với Sharding trên khóa shard (khoảng/băm; query router).
        \item MapReduce chỉ đọc trên các tập hợp (collection) lớn.
    \end{itemize}

    \paragraph{Key-Value Cache (Redis)}
    \begin{itemize}[leftmargin=*, label={--}]
        \item Hash-map trong bộ nhớ (In-memory) cho tốc độ đọc cực nhanh.
        \item Hỗ trợ cấu trúc dữ liệu phức tạp (set, queue) một cách tự nhiên.
        \item Bền vững qua nhật ký append-only và snapshot.
        \item Mẫu Read-through cache; ứng dụng quản lý việc vô hiệu hóa cache.
        \item Sao chép Master-slave cho tính sẵn sàng cao.
    \end{itemize}

    \paragraph{Graph DB (Neo4j)}
    \begin{itemize}[leftmargin=*, label={--}]
        \item Mô hình đồ thị thuộc tính: nút, quan hệ, nhãn, thuộc tính.
        \item Truy vấn đường dẫn hiệu quả; duyệt độ dài thay đổi qua Cypher.
        \item Ngôn ngữ khai báo với ký hiệu mũi tên cho các mẫu (patterns).
        \item Tính năng doanh nghiệp: caching, clustering, sao chép master-slave.
        \item Thiết kế tập trung tối ưu cho tải công việc đồ thị.
    \end{itemize}

    \paragraph{Wide-Column Stores (BigTable/HBase/Cassandra)}
    \begin{itemize}[leftmargin=*, label={--}]
        \item Bản đồ đa chiều thưa được sắp xếp: row key, column info, phiên bản.
        \item Lưu trữ LSM-tree chuyển đổi ghi ngẫu nhiên thành I/O tuần tự.
        \item HBase trên HDFS; ZooKeeper để điều phối; vùng (region) theo khoảng khóa.
        \item Column families nhóm lưu trữ; bộ định danh cột (qualifier) định nghĩa động.
        \item Cassandra: Sao chép không leader kiểu Dynamo; băm nhất quán (consistent hashing).
        \item Chiến lược nén (vd: cửa sổ thời gian); chỉ mục phụ SASI.
    \end{itemize}

    \paragraph{Data Warehouse Engines}
    \begin{itemize}[leftmargin=*, label={--}]
        \item HiveQL trên Hadoop; biên dịch sang kế hoạch thực thi MapReduce/Tez/Spark.
        \item SerDe hiển thị tập tin thô dưới dạng bảng; đẩy vị từ (predicate pushdown).
        \item Định dạng cột ORC/Parquet giảm thiểu I/O cho phân tích.
        \item Cloud OLAP: Snowflake/BigQuery/Redshift cho phân tích tương tác.
        \item Thực thi song song với lưu trữ mở rộng.
    \end{itemize}

    \paragraph{Khung xử lý (Processing Frameworks)}
    \begin{itemize}[leftmargin=*, label={--}]
        \item MapReduce: map/shuffle/reduce trên HDFS/HBase; chịu lỗi thông qua thực thi lại.
        \item Spark: RDD trong bộ nhớ; động cơ thống nhất (SQL, đồ thị, ML, streaming).
        \item Spark trên YARN; đọc từ HDFS/HBase.
        \item Tez: Thực thi DAG; pipeline các giai đoạn để tránh hiện thực hóa xuống HDFS.
        \item Nền tảng cho các hệ thống cấp cao hơn như Hive.
    \end{itemize}

    \subsection{Xử lý Luồng (Streaming)}
    \begin{itemize}[leftmargin=*, label={--}]
        \item \textbf{Mục đích:} Xử lý thời gian thực, không giới hạn cho phát hiện gian lận/xâm nhập, theo dõi, giám sát.
    \end{itemize}
    \paragraph{Thông điệp: Kafka Durable Log \& CDC}
    \begin{itemize}[leftmargin=*, label={--}]
        \item \textbf{Lưu trữ Log:} Đĩa append-only; producer thêm vào, consumer đọc tuần tự.
        \item \textbf{Phân hoạch/Thứ tự:} Phân hoạch Topic với thứ tự toàn cục trên mỗi phân hoạch qua offset.
        \item \textbf{Độ bền/Replay:} Retention giữ lại các bộ; việc tiêu thụ là chỉ đọc; consumer có thể replay từ offset.
        \item \textbf{CDC transport:} Luồng Change Data Capture bảo toàn thứ tự; DB nguồn là leader, follower xây lại trạng thái.
        \item \textbf{Nén Log:} Giữ lại lần ghi cuối cùng cho mỗi khóa, cho phép snapshot bảng mới nhất.
    \end{itemize}
    \paragraph{Tính toán: Flink/Spark \& Cloud}
    \begin{itemize}[leftmargin=*, label={--}]
        \item \textbf{Apache Flink:} Động cơ streaming thực thụ; thực thi pipeline; checkpoint để phục hồi.
        \item \textbf{Spark Streaming:} Microbatching (\~1s) trên RDD; tính toán lại khi lỗi.
        \item \textbf{Cloud:} Kinesis (log-based), Dataflow (checkpointed pipelines), Azure Stream Analytics (SQL streaming được quản lý).
    \end{itemize}
    \paragraph{Cửa sổ (Windows): Tumbling vs Hopping}
    \begin{itemize}[leftmargin=*, label={--}]
        \item \textbf{Tumbling (Trượt không chồng):} Độ dài cố định, liền kề, không chồng lấp (vd: 1 phút).
        \item \textbf{Hopping (Trượt chồng lấp):} Độ dài cố định, có chồng lấp (vd: cửa sổ 5 phút trượt mỗi 1 phút).
        \item \textbf{Ngữ nghĩa thời gian:} Thời gian sự kiện (Event time) vs Thời gian xử lý (Processing time); quản lý sự kiện đến muộn.
    \end{itemize}
    \paragraph{Stream Joins: Có giới hạn thời gian \& Có trạng thái}
    \begin{itemize}[leftmargin=*, label={--}]
        \item \textbf{Stream-Stream:} Kết nối cửa sổ trên khóa có giới hạn thời gian (vd: trong vòng 30 phút).
        \item \textbf{Stream-Table:} Làm giàu sự kiện dùng một quan hệ/changelog được xem như bảng.
        \item \textbf{Table-Table:} Kết nối các changelog để duy trì khung nhìn vật lý hóa (vd: tweets $\times$ follows).
        \item \textbf{Trạng thái (State):} Bộ xử lý duy trì trạng thái theo khóa và bộ đệm cửa sổ để khớp các sự kiện đến.
    \end{itemize}
    \paragraph{Ngữ nghĩa Exactly-Once}
    \begin{itemize}[leftmargin=*, label={--}]
        \item \textbf{Mục tiêu:} Đầu ra tương đương với việc thực thi không lỗi (không mất/không trùng).
        \item \textbf{Phục hồi Framework:} Microbatching (Spark) \& checkpointing (Flink) khởi động lại từ điểm nhất quán; loại bỏ đầu ra một phần.
        \item \textbf{Giao dịch phân tán:} Atomic commits cho trạng thái + thông điệp trong bộ xử lý.
        \item \textbf{Tính lũy đẳng bên ngoài:} Dùng offset/khóa duy nhất, bền vững để làm các tác dụng phụ trở nên lũy đẳng (khử trùng khi ghi).
    \end{itemize}

    \subsection{Pipelines: ETL vs ELT}
    \begin{itemize}[leftmargin=*, label={--}]
        \item \textbf{ETL:} Extract $\rightarrow$ Transform $\rightarrow$ Load (biến đổi trước khi nạp).
        \item \textbf{ELT:} Extract $\rightarrow$ Load $\rightarrow$ Transform (dùng khả năng tính toán của kho dữ liệu; schema-on-read).
        \item \textbf{Kafka:} Transport/CDC stream cho Extract/Load.
        \item \textbf{Airflow:} Điều phối các phụ thuộc batch cho Transform/Load.
    \end{itemize}

    \subsection{IoT}
    \begin{itemize}[leftmargin=*, label={--}]
        \item \textbf{Nguồn:} Cảm biến/RFID (tốc độ cao, dữ liệu tín hiệu đa dạng).
        \item \textbf{Ứng dụng:} Nông nghiệp thông minh, giám sát vận động viên, theo dõi di chuyển.
        \item \textbf{Giao thức:} MQTT cho nạp dữ liệu thời gian thực.
    \end{itemize}

    \subsection{So sánh}
    \paragraph{MongoDB vs Redis vs Neo4j}
    \begin{itemize}[leftmargin=*, label={--}]
        \item \textbf{MongoDB:} Document store, BSON, replica sets, sharding, MapReduce.
        \item \textbf{Redis:} In-memory key-value/cache, bền vững, sao chép; đọc nhanh.
        \item \textbf{Neo4j:} Mô hình đồ thị, Cypher cho đường dẫn, clustering; truy vấn đồ thị hiệu quả.
    \end{itemize}
    \paragraph{Cassandra vs Hive vs Snowflake}
    \begin{itemize}[leftmargin=*, label={--}]
        \item \textbf{Cassandra:} Wide-column, sao chép không leader, nhất quán cuối cùng, LSM + compaction, SASI.
        \item \textbf{Hive:} SQL-on-Hadoop biên dịch sang MapReduce/Tez/Spark; SerDe cho ORC/Parquet.
        \item \textbf{Snowflake:} Cloud DW OLAP; tương đương BigQuery/Redshift.
    \end{itemize}
    \paragraph{BigTable vs BigQuery}
    \begin{itemize}[leftmargin=*, label={--}]
        \item \textbf{BigTable:} Lưu trữ cột rộng phân tán; bản đồ thưa được sắp xếp; truy vấn khoảng; cảm hứng cho HBase.
        \item \textbf{BigQuery:} Phân tích OLAP tương tác; dòng dõi Dremel; động cơ SQL ở quy mô web.
    \end{itemize}

    \subsection{Kho dữ liệu (Data Warehousing)}
    \begin{itemize}[leftmargin=*, label={--}]
        \item \textbf{Đặc điểm:} Hướng chủ đề, tích hợp, bất biến, biến thiên theo thời gian; ít truy vấn nhưng quét lớn.
        \item \textbf{Lược đồ:} Bảng Fact (độ đo + khóa chiều) và bảng Dimension (chiều); Hình sao (Star) vs Bông tuyết (Snowflake).
        \item \textbf{Lưu trữ:} Ưu tiên hướng cột; Teradata, Sybase IQ, Redshift; Oracle/HANA/SQL Server hỗ trợ cột.
        \item \textbf{Tích hợp Big Data:} Hadoop với Hive/Spark SQL cho SQL trên tập tin phân tán.
    \end{itemize}

    \subsection{Mô hình Đa chiều}
    \begin{itemize}[leftmargin=*, label={--}]
        \item \textbf{Dạng bảng (Dimensional):} Lược đồ Star/Snowflake cho phép phân tích chiều.
        \item \textbf{Data Cube:} \texttt{GROUP BY CUBE} (mọi tập con); \texttt{ROLLUP} (tập con phân cấp).
    \end{itemize}

    \subsection{DW vs DL vs Lakehouse}
    \begin{itemize}[leftmargin=*, label={--}]
        \item \textbf{DW (Kho dữ liệu):} Schema-on-write; nhất quán mạnh; dữ liệu có cấu trúc; OLAP (ETL).
        \item \textbf{DL (Hồ dữ liệu):} Schema-on-read; dữ liệu thô đa định dạng; lưu trữ rẻ; Hadoop/Spark để truy vấn.
        \item \textbf{Lakehouse:} Lai ghép, kết hợp sự linh hoạt của hồ với quản lý của kho (ACID, thực thi lược đồ, đánh chỉ mục).
    \end{itemize}
\end{multicols}
\end{document}